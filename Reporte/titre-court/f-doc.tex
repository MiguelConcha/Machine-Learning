\PassOptionsToPackage{english}{babel}
\documentclass[11pt,french]{smfart}

\usepackage[english]{babel}


\usepackage[utf8]{inputenc} % Required for inputting international characters
\usepackage{stix} % Use the STIX fonts
\usepackage{amssymb,url,xspace,smfthm}

\makeatletter
    \def\ps@copyright{\ps@empty
    \def\@oddfoot{\hfil\small 2 de marzo de 2018}}
\makeatother





\title{Sistema de clasificación automática de críticas de películas}
\date {\today}
\author{Facultad de Ciencias, UNAM\\ Profesor Gustavo de la Cruz Fuentes}



\begin{document}

\selectlanguage{english}
%----------------------------------------------------------------------------------------
%	TITLE PAGE
%----------------------------------------------------------------------------------------

\begin{titlepage} % Suppresses displaying the page number on the title page and the subsequent page counts as page 1
	
	\raggedleft % Right align the title page
	
	\rule{1pt}{\textheight} % Vertical line
	\hspace{0.05\textwidth} % Whitespace between the vertical line and title page text
	\parbox[b]{0.75\textwidth}{ % Paragraph box for holding the title page text, adjust the width to move the title page left or right on the page
		
		{\Huge\bfseries Reconocimiento de \\[0.4\baselineskip] Patrones y Aprendizaje  \\[0.4\baselineskip] Automatizado}\\[2\baselineskip] % Title
		{\large\textit{Clasificación de Documentos}}\\[3\baselineskip] % Subtitle or further description
		{\Large\textsc{M. Concha Vázquez} \\[0.1\baselineskip] \textsc{L. Hernández Cano} \\[0.1\baselineskip] \textsc{M.X. Lezama Hernández} \\[0.1\baselineskip] \textsc{E. Vázquez Salcedo} } % Author name, lower case for consistent small caps
		
		\vspace{0.5\textheight} % Whitespace between the title block and the publisher
		
		{\noindent Facultad de Ciencias, Univerisad Nacional Autónoma de México.}\\[\baselineskip] % Publisher and logo
		
	}

\end{titlepage}

%----------------------------------------------------------------------------------------


%\def\smfbyname{}

\begin{abstract}

\end{abstract}


\maketitle

\tableofcontents

\section{Introduction}





\subsection{\'Enonc\'e g\'en\'erique}



\subsection{Autres \'enonc\'es}



\section{Adapter un manuscrit depuis un autre dialecte}


\subsection{Depuis une classe \LaTeXe}


\subsection{Depuis \LaTeX2.09}

\begin{thebibliography}{99}

\end{thebibliography}



\end{document}






